\chapter{Summary}

% The review
% – brief summary (2-3 lines)
%   – “If you can’t, there is probably something wrong with the paper” [ACM CHI FAQ]
% – Contribution 
%   - what is new? is it significant? (novelty/contribution)
%   - would it stimulate further work? (impact)
%   - how relevant is it to the community? (relevance)
% – Quality of the research
%   - how sound is the work?
%   - how appropriate/reliable are the methods used?
%   - how reasonable are the interpretations?
%   - how does it relate to existing work?
%   - can an experienced practitioner in the field duplicate the results?
% – Quality of the writing
%   - outline, language, spelling, grammar, figures, ...
% –  Recommend acceptance / rejection

The World Wide Web organizes information in HTML documents, which hold semi-structured data. For a server generated document from a template, information schema can be implied, and structured data extraction can be automated with queries, i.e. (web) wrappers. When user interface gets updated, the document structure changes, and wrapper has a tendency to break. Extracting attributes from multiple data records in a robust way from a single web page is the subject of this project. Current state of the art methods allow to detect template-based data regions on the page, to extract data from a single region in a robust way, and repeatedly extract data from multiple regions. In this thesis, we combine the three ideas into a new method for building a robust record-level wrapper from a single user-annotated web page. We design and implement the algorithm and run empirical tests. Experimental results using a large number of web pages from multiple domains show that the proposed approach works with a high precision and within reasonable execution time on commodity hardware.
