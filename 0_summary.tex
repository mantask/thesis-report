\chapter{Summary}
\label{summary}

% The review
% – brief summary (2-3 lines)
%   – “If you can’t, there is probably something wrong with the paper” [ACM CHI FAQ]
% – Contribution 
%   - what is new? is it significant? (novelty/contribution)
%   - would it stimulate further work? (impact)
%   - how relevant is it to the community? (relevance)
% – Quality of the research
%   - how sound is the work?
%   - how appropriate/reliable are the methods used?
%   - how reasonable are the interpretations?
%   - how does it relate to existing work?
%   - can an experienced practitioner in the field duplicate the results?
% – Quality of the writing
%   - outline, language, spelling, grammar, figures, ...
% –  Recommend acceptance / rejection

The World Wide Web organizes information in semi-structured HTML documents. For a template-based web page that contains a list of items, information schema can be implied and structured data can be extracted with a query, i.e. a (web) wrapper. When a user interface gets updated, the document structure changes, and the wrapper has a tendency to break. 

Extracting structured data from multiple data records in a robust way from a single web page is the subject of this project. Current state-of-the-art methods allow to detect template-based data regions on a page, to extract data from a single data record in a robust way, and repeatedly extract data from multiple records. 

In this thesis, we have combined the three ideas into a new method, which deals with building a robust record-level wrapper from a single user-annotated web page. We have designed, implemented, and tested the algorithm. Experimental results using a large number of web pages from multiple domains show, that the proposed approach works with high precision and within reasonable execution time on commodity hardware.


% vim:wrap linebreak nolist:
