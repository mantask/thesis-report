\chapter{Robust record-level web extraction algorithm}
% Describe your own work (how you reached your goal) and take care to motivate your choices. Don't just describe all the things you did – tell us why.

HTML has a tree structure and can be viewed as an ordered labled tree.

Step 1 of the algorithm on a high-level:

1. Build probabilistic wrapper to locate candidate node in a new tree.\\
2. Build a regional wrapper to locate candidate node inside single record.\\
   - Locate data records in original tree (that contain dist node)\\
   - Build a generic record tree by merging all data records inside original tree.\\
   - Create a regional wrapper for data record wrapping\\

Step 2 of the algorithm:

1. Locate candidate node inside new page.\\
2. Find data regions inside new page (with candidate node).\\
3. Match data records with regional wrapper.\\


% Define the content more carefully: all sections and a brief description what you will write in each of them. Define the main concepts you will need and fix the notations. Then you can write the chapters in any order you want. Make also a work plan: what you will do and when.

% Specify your topic carefully. Don’t take too large topic!  Invent a preliminary title for your thesis and define the content in a coarse level (main chapters). Ask your supervisor’s approval! Decide with your supervisor what material you should read or what experiments to make.  

% You can write the thesis after you have read all material or made all experiments. However, you can begin to write some parts already when you are working. Often you have to change your design plan, but it is just life! Ask feedback from your supervisor, when your work proceeds.

\IncMargin{2em}
\begin{algorithm}
	\KwData{$w, w', d(w)$}
	\KwResult{$\{v'_i\}$}
	\DontPrintSemicolon
	\SetKwFunction{RegionBoundary}{RegionBoundary}
	\SetKwFunction{ProbWrapper}{ProbWrapper}
	\SetKwFunction{RecordRegions}{RecordRegions}
	\BlankLine

	$v_{boundary}$ $\leftarrow$ \RegionBoundary{$w, d(w)$}\;
	$\phi_{boundary}$ $\leftarrow$ \ProbWrapper{$w, v_{boundary}$}\;
	$\{r_{i}\}$ $\leftarrow$ \RecordRegions{$w, v_{boundary}$}\;
	$\phi_{region}$ $\leftarrow$ \ProbWrapper{$r_1, d(w)$}\;
	$\{r'_{i}\}$ $\leftarrow$ \RecordRegions{$w', \phi_{boundary}(w')$}\;
	return $\{ \phi_{region}(r'_i) \}$

	\caption{Find record vertexes}
\end{algorithm}
\DecMargin{2em}

\IncMargin{2em}
\begin{algorithm}
	\KwData{$w, v$}
	\KwResult{$v_{boundary}$}
	\DontPrintSemicolon
	\SetKwFunction{RecordRegions}{RecordRegions}
	\BlankLine

	\For{$v_i \in siblings(v)$}{
		ref [Liu'03]
	}

	\caption{Finding region boundary}
\end{algorithm}
\DecMargin{2em}

% vim: set wrap
